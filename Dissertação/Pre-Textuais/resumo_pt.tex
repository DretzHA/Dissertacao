\begin{resumo}
Entre as diversas técnicas de radiofrequência utilizadas para obter localização, como o RSSI, Tempo de Voo e Ângulo de Chegada, que são combinadas com algoritmos para determinar o posicionamento de um alvo em ambientes internos, o AdC vem ganhando interesse desde sua incorporação no \textit{Bluetooth Low Energy} 5.1, devido à sua precisão, baixo consumo de energia, custo reduzido e facilidade de implementação. Este trabalho visa reduzir o erro estimado de posição de um alvo por meio de multilateração baseada em RSSI, bem como a posição estimada usando Ângulo de Chegada e RSSI empregando filtros estocásticos, e propor um sistema com precisão aprimorada usando a saída filtrada combinada de ambas as técnicas. Assim, será possível ter um sistema de posicionamento interno BLE 5.1 mais preciso e eficiente em comparação com outras técnicas utilizadas, que pode ser usado para localizar pessoas e ativos em ambientes internos com baixo erro de estimativa. Para alcançar esse objetivo, este trabalho utilizará um banco de dados de medições reais de RSSI e Ângulo de Chegada de um nó alvo BLE 5.1 e um conjunto de antenas em um ambiente de 14x8m, fornecido pela comunidade acadêmica, e aplicará filtros estocásticos aos algoritmos de localização do Ângulo de Chegada e RSSI para reduzir o erro de estimativa. Os erros serão calculados usando o RMSE, e os resultados serão comparados com os da literatura. Espera-se alcançar alta precisão na estimativa da localização de um alvo usando RSSI e Ângulo de Chegada e o sistema combinado proposto alcançará um erro de localização menor do que sistemas separados. A técnica de localização interna apresentada neste trabalho pode ser usada para substituir sistemas de localização de alto custo ou baixa precisão em várias aplicações em indústrias, saúde, comércio, logística e outras áreas.
\end{resumo}